\cvsection{Experience}
\begin{cventries}
\cventry
    {Full time remote role} % Short desc
    {Senior Aerodynamicist} % Title
    {Aurora Flight Sciences, a Boeing Company} % Where
    {Sept 2021 to Present} % Date
    {S\&C and Performance Senior Engineer on \href{www.wisk.aero}{www.wisk.aero}:\vspace{.5cm}
    \begin{cvitems}
    	\item Successfully developed methodology to quantify safety differences between competing control surface designs for eVTOL aircraft.
    	\item Designed control surface geometry - reduces separation vs. clean wing due to throughflow.
    	\item Demonstrated aircraft design meets MIL-STD-8785C or MIL-STD-1797B criteria for manoeuvrability
    	\item Created workflow to get clean foils/control surfaces into wing loft - maintaining curvature continuity and eliminating manual smoothing by lofter.
    \end{cvitems}
    \vspace{.5cm}
    S\&C Lead on Virgin Galactic Mothership venture:\vspace{.5cm}
    \begin{cvitems}
    	\item Demonstrated deficiencies in existing control system and led small modifications to improve agility by 50\%.
    	\item Led team of Aurora engineers in night-shift working to quantify uncertainties in actual aircraft geometry (crawling around the inside of VMS Eve).
    \end{cvitems}
	\vspace{.5cm}
    Others:\vspace{.5cm}
    \begin{cvitems}
    	\item Wrote \texttt{scToolbox.py} - a module to gain estimates of flying/handling qualities at preliminary design stage.
     	\item Raised safety concerns on aircraft program that led to Chief Engineer intervention.
    \end{cvitems}
    }
%\vspace{1cm}


 \cventry
    {Tier II Faculty role} % Short desc
    {Industry Assistant Professor} % Title
    {Illinois Institute of Technology, Chicago IL} % Where
    {March 2020 to Aug 2021} % Date
    {%\hspace{-.5cm}Taught MMAE 410 Flight Mechanics:\vspace{.5cm}
    \begin{cvitems}
    	\item Created research programme with industrial partners - AIAA Journal Paper Submitted, second being written.
    	\item Developed fully-online teaching methodologies in response to Covid-19 - moved to HQ video lectures - see \textcolor{cyan}{\textbf{\href{https://youtu.be/ZxYc9ORAGtw}{youtu.be/ZxYc9ORAGtw}}}
    \end{cvitems}
    }
%\vspace{1cm}
  \cventry
    {Faculty role teaching and supervising undergraduate students.} % Short desc
    {Aerospace Lecturer} % Place
    {\hspace{-3cm}} % Where
    {August 2018 to March 2020} % Date
    {%\hspace{-.5cm}Taught MMAE 410 Flight Mechanics:\vspace{.5cm}
    \begin{cvitems}
    	\item MMAE 410 Flight Mechanics - wrote new syllabus, and accompanying website in lieu of textbook - \textcolor{cyan}{\textbf{\href{www.aircraftflightmechanics.com}{aircraftflightmechanics.com}}}.
    	\item MMAE 315 Aerospace Laboratory - taught fundamentals of experimental methods, drawing from own industrial experience
    	\item MMAE 304 Mechanics of Aerostuctures, developed a new syllabus; all lectures \textcolor{cyan}{\textbf{\href{https://www.youtube.com/watch?v=XaKwlHxDsKE&list=PLVyHCaFIZQV_Fnchp_8yxVAGedKU8rDGZ}{available online}}}.
    	%\item Proposed, developed, and delivered new syllabus for MMAE 100 Introduction to the Profession; taught as a course to aerospace majors, gave students grounding in aeronautical fundamentals such as aircraft anatomy, basic aircraft performance, and fundamentals of propulsion. %Course adapted from an original proposal for a Sophomore-level course, written to facilitate a greater depth of material in MMAE 410 Flight Mechanics (that is, 410 is an \textsl{enormous} course, and sufficient depth cannot be achieved in a single semester - some can easily be taught at an earlier level).
%    	\item Led two students in Summer 2019 in development of `adaptive blade element code', as proof-of-concept of modelling technique. Students continuing work as MMAE 594 in 2020.
    \end{cvitems}
    \hspace{-.5cm}%Received excellent feedback from students in all classes.
    }
    

  
  \setlength{\itemsep}{10pt}


    
    


\end{cventries}
