
\begin{cventries}

\vspace{.5cm}
\cventry
    {Senior role within Aerodynamics and Test Techniques group} % Short desc
    {Senior Aerodynamics Engineer} % Place
    {\hspace{-2cm}Aircraft Research Association (ARA), Bedford UK} % Where
    {December 2016 to June 2018} % Date
    {Technical Authority for Rotary Testing:\vspace{.5cm}
    \begin{cvitems}
    	\item Developed Pressure-Sensitive Paint (PSP) and Blade Deformation Measurement systems for use on 3ft diameter propeller at 6000RPM.
    	\item Wrote data reduction scripts for live Fourier analysis of rotating-frame data for conversion into inertially-fixed forces.
    	\item Lead in the investigation of, and mitigation against electromagnetic interference issues encountered during previous testing.
%    	\item Developed new wind-tunnel correction methods for propulsive testing in a perforated-wall tunnel.
    \end{cvitems} 
    \vspace{.5cm}
    Created \textsl{DynAMoS} - `DYNamic Analysis of MOdel Stresses' for safe wind-tunnel testing involving dynamic behaviour:
    \vspace{.5cm}
    \begin{cvitems}
    	\item Live feedback system to monitor dynamic stresses in aircraft model components, separating dynamic and steady-state loads.
%    	\item Modified-Goodman approach ensures fatigue is avoided whilst 
    \end{cvitems} 
    \vspace{.5cm}    
    Successfully implemented, trialled and sold Model Deformation Measurement (MDM) system to a major US Civil Airframer:
    \vspace{.5cm}
    \begin{cvitems}
%    	\item Utilising stereoscopic Digital Image Correlation (DIC) and a stochastic wing marker pattern, the deflected or `hot' surface profile of a wind tunnel model can be determined with accuracy $<$0.1mm.
    	\item Fully-mapped data back to aircraft global reference system for production of modified CAD surfaces to implement in CFD.
%    	\item System recently extended to capture dynamic half-model wing bending in response to gust.
    \end{cvitems} 
    \vspace{.5cm}
        Technical lead for balance data reduction:
    \vspace{.5cm}
    \begin{cvitems}
    	\item Created tools for conversion between NLR and ARA wind tunnel balance matrices.
    	\item Technical point of contact between ARA and Triumph (Calspan), and Aerophysics Research Instruments.
    \end{cvitems}       
%     \vspace{.5cm}
%             Created `STAR' system - Safely Testing ARA Rotors:
%    \vspace{.5cm}
%    \begin{cvitems}
%    	\item Controls rotor systems in TWT, maintaining safe pitch angle and RPM based on tunnel speed to avoid stall flutter.
%   	\end{cvitems}       
     \vspace{.5cm}       
  Created `DRAFT NP' system - captures n-per-revolution harmonic data from transducers based in the rotating reference frame:
  \vspace{.5cm}
         \begin{cvitems}
    	\item Developed from codes written during IMPACTA testing (see overleaf) - system enables determination of inertially-fixed parameters from rotating-frame data.
   	\end{cvitems}       
     \vspace{.5cm}        
    }




%    \vspace{-.1cm}

  \cventry
    {Experimental Aerodynamics dept. and then Aerodynamic Capability dept.} % Short desc
    {Aerodynamics Engineer/Project Supervisor} % Place
    {\hspace{-2cm}ARA} % Where
    {\hspace{-2cm}September 2014 to December 2016} % Date
    {Initially employed at ARA to bring specialist rotary aerodynamics knowledge based on PhD findings, in addition to being able to lead large wind tunnel programmes. Notable achievements include:\newline\vspace{.1cm} Designed bespoke dynamic data acquisition/reduction for IMPACTA programme (�2m wind tunnel test for GE Aviation):
   \vspace{.5cm}
    \begin{cvitems}
    	\item During testing, the existing NP load measurement system did not satisfy requirements and testing was halted. Took sole responsibility to measure Rotary Shaft Balance (RSB) data at 40MHz, and produced a frequency-domain data reduction suite. \textsl{Testing was continued at ARA based only on the system that I wrote}.
    	\item {ARA Representative for Vertical Lift Network (VLN)} - represented the UK's leading industrial wind tunnel capability at the VLN, a group comprising rotorcraft manufacturers and academic institutions.
    \end{cvitems} 
    }
    
%    \vspace{1cm}
  \cventry
    {Supplementary role as international business development representative in The Americas} % Short desc
    {Business Development Manager} % Place
    {\hspace{-2cm}ARA} % Where
    {December 2016 to June 2018} % Date
    {In charge of business development for the Americas - >\textsterling10m tests following US visits to major aerospace suppliers including \textsl{Boeing}, \textsl{Northrop Grumman}, \textsl{Raytheon}, \textsl{Lockheed Martin}, \textsl{Textron}, \textsl{Triumph}, \textsl{Aurora Flight Sciences}, \textsl{Calspan}, \textsl{USAFRL}, and many others. 
    }

    
%    \vspace{.5cm}
%      \cventry
%    {University of Glasgow Aerospace Sciences Division} % Role
%    {Doctoral Researcher} % Place
%    {University of Glasgow} % Where
%    {April 2011 to September 2014} % When
%    {Ph.D. studentship to research propeller n-per-rev loading. Research culminated in a load estimation code for GE Aviation - commercial restrictions on publication.
%%    \vspace{.5cm}
%%      \begin{cvitems}
%%      \item Funded by General Electric, %salary topped up to Postdoctoral level to attract top candidate. 
%%		%\item {Static/Dynamic Rotary Aeroelasticity} - 
%%		 commercial restrictions on publication
%%%		\item Using \textsl{Vortex/Panel Methods}, %- created a validated propeller/aircraft flowfield prediction code.
%%%%		\item 
%%%		\textsl{Blade-Element Momentum Theory}, implemented a novel utilisation of BEMT for the propeller at incidence.
%%%		\item  Identified flaws in industry standard-practice `steady-state' assumptions.
%%%		\item Highlighted that this is an unsteady aerodynamic problem - industry standard-practice is to ignore these effects.
%%%      \item In addition to my own research, I was responsible for teaching undergraduate laboratories, tutorials and for marking undergraduate coursework - including MATLAB assignments for Numerical Methods classes. \vspace{-.2cm}
%%            \end{cvitems}
%     }
     
    \vspace{1cm}
       \cventry
    {Flight Mechanics Dept., Helicopter System Design} % Role
    {Postgraduate Flight Mechanics Engineer} % Place
    {\hspace{-4cm}AgustaWestland (now Leonardo Helicopters), Yeovil} % Where
    {September 2010 to May 2011} % When
    {Selected by HSD Group because of knowledge in \textsl{Flight Mechanics} and \textsl{Aeroelasticity}. Role included
%    \vspace{.5cm}
%\begin{cvitems}%
validation of Flight Test data for AW159 Wildcat. Created a new set of software routines to analyse transient Handling Quality Data automatically.
%\item Assisted in creation of new AICAM/CAMRAD2 model for AW139.
%\end{cvitems}
 }

    \vspace{1cm}
\cventry 
    {Six-month industrial placement for Master's Project} % Role
    {Systems/Aerospace Engineer} % Place
    {\hspace{-5cm}Thales Land and Joint, Bury St. Edmunds} % Where
    {January to June 2010} % When
    {Industrially-based dissertation - ``\textsl{Review of Aerodynamic Load Calculation on the Vicon-18 Series Reconnaissance Pods}'':
%    \vspace{.5cm}
%    \begin{cvitems}
%  	\item Produced software to automate an ESDU-based method that was previously performed manually.
%  	\item Interface with network-based data to update mission parameters, creating a \textsl{live} load estimation tool for Thales based on current requirements. \vspace{-.4cm}
%    \end{cvitems}
    }
     
\end{cventries}
