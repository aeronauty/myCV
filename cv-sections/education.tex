\cvsection{Education}

%----------------------------------------------------------------------------------------
%	SECTION CONTENT
%----------------------------------------------------------------------------------------
\vspace{.25cm}
\begin{cventries}

%------------------------------------------------

\cventry
	{Dissertation: `Engineering Models of Aircraft Propellers at Incidence'} % Degree
	{Ph.D. Aerospace Engineering} % Institution
	{University of Glasgow} % Location
	{2011-2014} % Date(s)
	{\begin{cvitems}
	    	\item A combination of commercial and security classification precluded publication of key findings.
	    	\item New validation data (UoG) has been produced in late 2020 for validation of some results, with aim to publish two journal papers in 2021-22.
	  	\end{cvitems}
	}

\cventry
{Concentrations: Rotary Aerodynamics, Aeroelasticity, Turbomachinery, CFD}
{M.Eng (Hons) Aeronautical Engineering}
{University of Glasgow}
{2005-2010}{
{Final Grade: 1\textsuperscript{st}-class honours. \textbf{Graduated top of a class of twenty}.}}

%------------------------------------------------

\end{cventries}

%\begin{cventries}
%  \cventry
%    {Faculty role teaching and supervising undergraduate students.} % Short desc
%    {Aerospace Lecturer} % Place
%    {\hspace{-2cm}MMAE Dept., Illinois Institute of Technology, Chicago} % Where
%    {August 2018 to present} % Date
%    {%\hspace{-.5cm}Taught MMAE 410 Flight Mechanics:\vspace{.5cm}
%    \begin{cvitems}
%    	\item MMAE 410 Flight Mechanics - I wrote new syllabus to move away from a textbook-based approach, and wrote my own 260-page accompanying notes, with interactive MATLAB-based problems for students. 
%    	\item MMAE 315 and 319 Aerospace/Mechanical Laboratory - developed syllabus to teach students in the fundamentals of experimental methods, drawing from my own industrial experience.
%    	\item MMAE 304 Mechanics of Aerostuctures, developed a new syllabus to teach students the \textsl{method of virtual work}, and giving coursework in writing Finite-Element codes in MATLAB, and Python.
%    \end{cvitems}
%    \hspace{-.5cm}%Received excellent feedback from students in all classes.
%    }